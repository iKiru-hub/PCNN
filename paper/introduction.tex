\section{Introduction}
\hfill \break
\vspace {0.5cm}

% --- COGNITIVE MAP ---
% - intution
% - history
% - behaviour
% - previous models
During navigation, animals dynamically create rich representations of the environment, forming personalized cognitive maps.
The hippocampal area CA1 features spatial cells that adapt based on behavior and internal states. Computational models have usually obtained spatial tuning by training a deep
recurrent network for solving navigation tasks such as path integration \cite{sorscherUnifiedTheoryOrigin2019, cuevaEmergenceGridlikeRepresentations2018, baninoVectorbasedNavigationUsing2018}, lasting multiple numerous epochs and using backpropagation.
However, these training methods do not closely
align with real-time local learning paradigms used by animals. \\ In this study, it is introduced a rate model that generates place cells in one-shot as the agent navigates the environment by simply assigning the current spatial observation to a selected neuron while ensuring a sparse representation
(\textit{i.e.} spaced place fields).

% --- NEUROMODULATION ---
% - what it is
% - why it is relevant for cognitive maps
An important ingredient for the learning dynamics of our model is neuromodulation.
Neuromodulation is an important ingredient for biological neuronal dynamics, with different molecules covering a wide range of functions.
Previous models \cite{brzoskoRetroactiveModulationSpike2015, brzoskoSequentialNeuromodulationHebbian2017} inspired by experimental results crafted a simple spiking plasticity rule for reward-directed navigation where acetilcholine mediates explorative behaviour and dopamine reinforces memory of reward locations.
Other approached using deep artificial networks have applied neuromodulation in conjuction with other training practices, such as dropout probability \cite{meiEffectsNeuromodulationinspiredMechanisms2023}.
In this work, modulators are described as synaptic resources that are consumed by plasticity events, and their dynamics are modelled as leaky integrators.
Further, acetilcholine is used to mediate the generation of new place fields, while dopamine mediates the slow remapping of the place centers in
conjunction with a reward signal.
The concentration of acetilcholine is affected by the presence of active neurons or by the occurrence of a weight update.
Dopamine, on the other hand, is influenced by the presence of a reward.

This model successfully creates a representation of visited areas and recurrent connections are defined among similarly tuned cells.
Importantly, plasticity hyper-parameters such as the equilibrium concentration
and decay time-constant of modulators influence the density of place cells, impacting the encoding of behaviorally relevant information \cite{bittnerBehavioralTimeScale2017}.

This network is then used to solve a goal-directed navigation task, where the agent is trained to reach a target location.
The agent is equipped with a policy that modulates the exploration behaviour and the decision-making process.

% --- GOAL ---
% - goals statement
% - what's wrong with previous models
% - the proposed direction

