\section{Introduction}

Survival in complex environments demands efficient navigational strategies. From desert ants to humans, successful wayfinding—navigating toward goals out of sight—relies on internal spatial representations, or \textit{cognitive maps} \cite{golledgeCognitiveMapsSpatial2000, epsteinNeuralSystemsLandmarkbased2014}. These maps enable flexible planning and decision-making beyond simple stimulus-response associations.

Cognitive map theories propose multiple strategies for spatial navigation, ranging from simple route learning to survey-based and graph-based models.introdu
Route learning stores paths as action-position pairs but struggles with scalability and generalization at intersections \cite{peerStructuringKnowledgeCognitive2021, chrastilCognitiveMapsCognitive2014, wernerModellingNavigationalKnowledge2000}.
Survey maps, grounded in Euclidean geometry, offer greater flexibility \cite{chrastilCognitiveMapsCognitive2014, gallistelComputationsMetricMaps1996} but sometimes contradict neural and behavioral data showing geometric distortions and topological biases \cite{peerFormatCognitiveMap2024, warrenNonEuclideanNavigation2019, wagnerComparingPsychophysicalGeometric2008, rothkegelJudgingSpatialRelations1998}.
Labeled graphs strike a balance—encoding landmarks and transitions in a topological network that supports vector-like operations, planning, and prediction \cite{meilingerNetworkReferenceFrames2008, wangLearningReinforcementLearn2017, schinaziHippocampalSizePredicts2013}.

Neural substrates of spatial representations reside in the hippocampus (HP) and entorhinal cortex (EC), where specialized cells—including grid, border, speed, and place cells—encode geometric and contextual variables \cite{sargoliniConjunctiveRepresentationPosition2006, kropffSpeedCellsMedial2015, solstadGridCellsPlace2006}. Place cells, particularly in CA1, anchor cognitive maps \cite{donatoHowYouBuild2023}, integrating converging inputs from grid cells, CA3, and lateral EC \cite{bushWhatGridCells2014, neherGridCellsPlace2017a, liModelingPlaceCells2019, bilashLateralEntorhinalCortex2023}.

Neuromodulation plays a critical role in shaping this circuitry. Dopamine and other modulators adjust synaptic strength, tune place fields, and encode novelty and reward-prediction errors \cite{lismanHippocampalVTALoopControlling2005, duszkiewiczNoveltyDopaminergicModulation2019,
schultzNeuralSubstratePrediction1997}. These projections reshape spatial tuning \cite{kempadooDopamineReleaseLocus2016, retailleauMichelinRedGuide2014, bittnerBehavioralTimeScale2017, kaufmanRoleLocusCoeruleus2020}, support novelty detection \cite{duszkiewiczNoveltyDopaminergicModulation2019}, and transmit prediction errors, particularly via lateral EC inputs \cite{igarashiFunctionalDiversityTransverse2014, itoFunctionalDivisionHippocampal2012}.
These mechanisms echo principles of reinforcement learning (RL), highlighting the role of neuromodulation in adapting spatial representations to behavioral relevance \cite{sheynikhovichLongtermMemorySynaptic2023, schultzNeuralSubstratePrediction1997}.

Computational models have captured individual components of this system.
Early work proposed that the hippocampus encodes spatial position and direction \cite{poucetSpatialCognitiveMaps1993}, while topological models based on route learning highlight scalability challenges \cite{wernerModellingNavigationalKnowledge2000}.
More recent approaches draw from predictive coding and reinforcement learning, including successor representations and the Tolman-Eichenbaum Machine, which generalize across spatial and relational tasks while mimicking biological activity patterns \cite{stoewerNeuralNetworkBased2023, decothiPredictiveMapsRats2022, whittingtonTolmanEichenbaumMachineUnifying2020}.
Path integration models trained on motion cues give rise to grid- and place-like tunings \cite{baninoVectorbasedNavigationUsing2018, sorscherUnifiedTheoryOrigin2019, cuevaEmergenceGridlikeRepresentations2018}. Others incorporate reward-driven Hebbian plasticity modulated by neuromodulators \cite{brzoskoNeuromodulationSpikeTimingDependentPlasticity2019}. Yet, few architectures unify these ingredients into a biologically grounded system that learns online and adapts flexibly to novelty.

In this work, we introduce a biologically inspired model of cognitive map formation that integrates place cell representations, neuromodulatory signals, and graph-based spatial reasoning.
The model constructs a topological map online, linking place cells along experienced trajectories and enriching them with scalar-valued modulatory signals.
These modulators form analog fields across the map \cite{sosaHippocampalSequencesSpan2024}, drive local Hebbian plasticity in response to sensory prediction errors \cite{aliPredictiveCodingConsequence2021, sheynikhovichLongtermMemorySynaptic2023, bonoLearningPredictiveCognitive2023}, and help maintain and adapt reward-based neural representations \cite{schultzDopamineRewardPrediction2016, inglisModulationDopamineAdaptive2021, toblerAdaptiveCodingReward2005, coolsChemistryAdaptiveMind2019, decothiPredictiveMapsRats2022}.

This architecture supports efficient goal-directed navigation without extensive offline training, leveraging spatial priors, online plasticity, and modulatory feedback. We demonstrate how this system adapts to environmental changes and how neuromodulation influences place field allocation and remapping \cite{milsteinBidirectionalSynapticPlasticity2021, fentonRemappingRevisitedHow2024}, linking cognitive flexibility to underlying physiological mechanisms.

The remainder of the paper is organized as follows: Section 2 details the model and experimental setup; Section 3 presents results; Section 4 discusses broader implications and future directions.
