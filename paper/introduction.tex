\section{Introduction}
\hfill \break
\vspace {0.5cm}




% --- catchy intro ---
Agency in a large environment is a challenging task for any organism. Productive exploration and effective goal-directed navigation are essential in this regard, and become even more crucial when target locations are out of sight. In this scenario, often labelled as \textit{wayfinding}, an adequate understanding of one's surroundings is necessary.

For this scope, one of the tools evolved by the mammalian brain are neural spatial representations of the environment, enriched by repeated experiences. In the literature, they are often referred to as cognitive maps, which where first introduced by Tolman who noticed the reward-driven navigation ability of rats in a water maze \cite{tolmanCognitiveMapsRats1948}.

% --- cognitive map and function ---
There are multiple approaches that can be used for solving the task of reaching a known target in its environment with a cognitive map. However, it is still debated which are the ones the brain preferred the most.
One of straighforward possibilities is \textit{route learning}, where the gist is to memoriezed the previous paths as simple action-position pairs. This approach is effective in small environments, but it is impractical for constructing a proper map to exploit more generally, given the ambiguity at combining intersecting paths \cite{peerStructuringKnowledgeCognitive2021}.
In contrast, a more general strategy is to rely on \textit{survey knowledge}, where the subject builds an explicitly general representation of the environment by endowing it with an Euclidean metric, namely a global coordinate system based on distances and angles in which perform vector operations.
The resulting map is highly flexible, and allows for the computation of short-cuts and arbitrary routes. However, it lack clear and convicing experimental evidence, with several studies in humans and mice highlighting how real brain maps often violate the strict algebraic constrains.
Another possibility is a so-called \textit{labeled graph}, which consists of a topological graph of the environment over locations in space. Importantly, it does not possess a universal metric but instead relies on local labels for supporting information about distances and angles.
The lack of a rigid operational structure leads to more tolerance for global spatial incosistencies, while still allowing for affine vector operations, at least locally.
Multiple studies have shown support for this type of representation, in which the graph-based map and its labels are generated and successfully exploited for goal-reaching tasks, despite clearly infringing Euclidean geometry.


% --- cognitive map and neuroanatomy ---
Regarding the main neural substrates of the cognitive maps, a variety of neuronal types has been associated to allocentric and egocentric spatial features.
Among others, border cells have being associated to boundary detection, speed cells with the magnitude of the perceived velocity, head-direction cells with the anglular position of the head as per vestibular perception, place cells with unimodal tuning for spatial positions, and grid cells with an hexagonal periodic tuning.
Parts of this neuronal ecosystem has been found in various part of the cortex, but more predominatly in the entorhinal cortex (EC), in particular the medial region (MEC), and sub-regions of the hippocampal formation, particularly the cornus ammonis area CA3 and CA1, and the subiculum (SB).
Specifically, place cells in CA1 have been often considered as the component of the cognitive maps.
The origin of the spatial tuning of these cells is still debated. Some theories trace it back to competitive dynamics over projections from entorhinal grid cells through the temporo-ammonic pathway, while others point at the Schraffer's collaterals from CA3.


% --- cognitive map and neuromodulation ---


% --- computational models ---


% --- problem statement ---


% --- proposed solution ---


% --- COGNITIVE MAP ---
% - intution
% - history
% - behaviour
% - previous models
During navigation, animals dynamically create rich representations of the environment, forming personalized cognitive maps.
The hippocampal area CA1 features spatial cells that adapt based on behavior and internal states. Computational models have usually obtained spatial tuning by training a deep
recurrent network for solving navigation tasks such as path integration \cite{sorscherUnifiedTheoryOrigin2019, cuevaEmergenceGridlikeRepresentations2018, baninoVectorbasedNavigationUsing2018}, lasting multiple numerous epochs and using backpropagation.
However, these training methods do not closely
align with real-time local learning paradigms used by animals. \\ In this study, it is introduced a rate model that generates place cells in one-shot as the agent navigates the environment by simply assigning the current spatial observation to a selected neuron while ensuring a sparse representation
(\textit{i.e.} spaced place fields).

% --- NEUROMODULATION ---
% - what it is
% - why it is relevant for cognitive maps
An important ingredient for the learning dynamics of our model is neuromodulation.
Neuromodulation is an important ingredient for biological neuronal dynamics, with different molecules covering a wide range of functions.
Previous models \cite{brzoskoRetroactiveModulationSpike2015, brzoskoSequentialNeuromodulationHebbian2017} inspired by experimental results crafted a simple spiking plasticity rule for reward-directed navigation where acetilcholine mediates explorative behaviour and dopamine reinforces memory of reward locations.
Other approached using deep artificial networks have applied neuromodulation in conjuction with other training practices, such as dropout probability \cite{meiEffectsNeuromodulationinspiredMechanisms2023}.
In this work, modulators are described as synaptic resources that are consumed by plasticity events, and their dynamics are modelled as leaky integrators.
Further, acetilcholine is used to mediate the generation of new place fields, while dopamine mediates the slow remapping of the place centers in
conjunction with a reward signal.
The concentration of acetilcholine is affected by the presence of active neurons or by the occurrence of a weight update.
Dopamine, on the other hand, is influenced by the presence of a reward.

This model successfully creates a representation of visited areas and recurrent connections are defined among similarly tuned cells.
Importantly, plasticity hyper-parameters such as the equilibrium concentration
and decay time-constant of modulators influence the density of place cells, impacting the encoding of behaviorally relevant information \cite{bittnerBehavioralTimeScale2017}.

This network is then used to solve a goal-directed navigation task, where the agent is trained to reach a target location.
The agent is equipped with a policy that modulates the exploration behaviour and the decision-making process.

% --- GOAL ---
% - goals statement
% - what's wrong with previous models
% - the proposed direction

