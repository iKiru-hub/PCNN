\section{Introduction}
\hfill \break
\vspace {0.5cm}




% --- catchy intro ---
Agency in a large environment is a challenging task for any organism. Productive exploration and effective goal-directed navigation are essential in this regard, and become even more crucial when target locations are out of sight. In this scenario, often labelled as \textit{wayfinding}, an adequate understanding of one's surroundings is necessary \cite{golledgeCognitiveMapsSpatial2000, epsteinNeuralSystemsLandmarkbased2014}.

For this scope, one of the tools evolved by the mammalian brain are neural spatial representations of the environment, enriched by repeated experiences. In the literature, they are often referred to as cognitive maps, which where first introduced by Tolman who noticed the reward-driven navigation
ability of rats in a maze \cite{tolmanCOGNITIVEMAPSRATS1948}.

% --- cognitive map and function ---
There are multiple approaches that can be used for solving the task of reaching a known target in its environment with a cognitive map, and the choice might depend on the task as well as personal traits, at least in humans \cite{weisbergCognitiveMapsPeople2018, schinaziHippocampalSizePredicts2013}.
However, which are the ones the brain generally prefers the most is still debated.
One of straighforward possibilities is \textit{route learning}, where the gist is to memorized the previous paths as simple action-position pairs.
This approach is effective in small environments, but it is impractical for constructing a proper map to exploit more generally, given the ambiguity at combining intersecting paths \cite{peerStructuringKnowledgeCognitive2021, chrastilCognitiveMapsCognitive2014, wernerModellingNavigationalKnowledge2000}.
In contrast, a more general strategy is to rely on \textit{survey knowledge}, where the subject builds an explicitly general representation of the environment by endowing it with an Euclidean metric, namely a global coordinate system based on distances and angles in which perform vector operations \cite{chrastilCognitiveMapsCognitive2014, gallistelComputationsMetricMaps1996}.
The resulting map is highly flexible, and allows for the computation of short-cuts and arbitrary routes. However, it lacks convicing experimental evidence, with several studies in humans and mice highlighting how real brain maps often violate the strict algebraic constrains
\cite{peerFormatCognitiveMap2024, warrenNonEuclideanNavigation2019, wagnerComparingPsychophysicalGeometric2008, rothkegelJudgingSpatialRelations1998}.
Another possibility is a so-called \textit{labeled graph} \cite{chrastilCognitiveMapsCognitive2014, meilingerNetworkReferenceFrames2008, wangLearningReinforcementLearn2017}, which consists of a topological graph of the environment over locations in space.
Importantly, it does not possess a universal metric but instead relies on local labels, built through experiences \cite{ishikawaSpatialKnowledgeAcquisition2006}, for supporting information about distances and angles.
The lack of a rigid operational structure leads to more tolerance for global spatial incosistencies, while still allowing for affine vector operations, at least locally.
Multiple studies have shown support for this type of representation, in which the graph-based map and its labels are generated and successfully exploited for goal-reaching tasks, despite clearly infringing Euclidean geometry \cite{byrneMemoryUrbanGeography1979, warrenNonEuclideanNavigation2019, schinaziHippocampalSizePredicts2013}.
This online formation of position nodes is also aligned with \textit{path integration}, the process of integrating velocity vectors for tracking the trajectory leading home, supported by observations in multiple species \cite{wehnerVisualNavigationInsects1996, gillnerNavigationAcquisitionSpatial1998} and theoretical analysis \cite{gallinaroSynapticWeightsThat2023}.

% --- cognitive map and neuroanatomy ---
Regarding the main neural substrates of the cognitive maps, a variety of neuronal types has been associated to allocentric and egocentric spatial features.
Among others, border cells have being associated to boundary detection, speed cells with the magnitude of the perceived velocity, head-direction cells with the anglular position of the head as per vestibular perception, place cells with unimodal tuning for spatial positions, and grid cells with an
hexagonal periodic tuning \cite{sargoliniConjunctiveRepresentationPosition2006, kropffSpeedCellsMedial2015, solstadGridCellsPlace2006}.
Parts of this neuronal ecosystem has been found in various part of the cortex, but more predominatly in the entorhinal cortex (EC), in particular the medial region (MEC), and sub-regions of the hippocampal formation, particularly the cornus ammonis area CA3 and CA1, and the subiculum (SB).
Specifically, place cells in CA1 have been often considered as the component of the cognitive maps \cite{donatoHowYouBuild2023}.
The origin of the spatial tuning of these cells is still debated. Some theories trace it back to competitive dynamics over projections from entorhinal grid cells through the temporo-ammonic pathway, while others point at the Schraffer's collaterals from CA3 \cite{bushWhatGridCells2014, neherGridCellsPlace2017a, liModelingPlaceCells2019, kubieSpatialFrequenciesGrid2015}.


% --- cognitive map and neuromodulation ---
\subsection{Neuromodulation of cognitive maps}
The upstream afferences to these place cells populations carry a variety of inputs, ranging from spatial information from the medial EC and CA3, to sensory and contextual data from the lateral EC \cite{bilashLateralEntorhinalCortex2023}.
In addition, they are also targeted by a multiple neuromodulators, molecules gating information and affecting neuronal dynamics in a multitude of ways, the most prominent of which in this context are dopamine and acetilcholine.

% dopamine
There are several set of dopaminergic innervations to the hippocampus, targeting all its main sub-components. The ones more directly involved with CA1 place cells are those from the ventral tegmental area (VTA) and locus coeruleus (LC) \cite{lismanHippocampalVTALoopControlling2005, tsetsenisActivationLocusCoeruleus2022, zhangExpectancyrelatedChangesFiring2024}.
Their actions have been linked to synaptic excitation, modulation of theglutamatergic projections from EC, and regulation of long term potentiation (LTP) through D1-like receptors \cite{edelmannDopaminergicInnervationModulation2018, wagatsumaLocusCoeruleusInput2018}.
One important function of dopamine concerning spatial navigation is the delivery of reward-related information, supporting memory consolidation of salient locations \cite{kempadooDopamineReleaseLocus2016, retailleauMichelinRedGuide2014} and active reshaping of the place cells tuning \cite{bittnerBehavioralTimeScale2017, kaufmanRoleLocusCoeruleus2020}.
Another extensively documented role is novelty detection \cite{duszkiewiczNoveltyDopaminergicModulation2019}, especially for contextual inputs from LEC \cite{wagatsumaLocusCoeruleusInput2018, igarashiFunctionalDiversityTransverse2014, itoFunctionalDivisionHippocampal2012}, as well as encoding prediction errors of reward events \cite{sheynikhovichLongtermMemorySynaptic2023, schultzNeuralSubstratePrediction1997}

% acetilcholine
The cholinergic innervation to the hippocampus is mainly provided by the medial septum (MS). Through the action of different muscarinic recptors, it modulates the afferences to CA1 from MEC and CA3, and promotes both long-term synaptic potentiation and depression \cite{fuchsbergerModulationHippocampalPlasticity2022, sugisakiCholinergicModulationSpike2011, dennisActivationMuscarinicM12016}.
Its role in spatial navigation includes the support of explorative behaviour, contributions to novelty detection, and attentional regulation by increasing the input signal-to-noise ratio \cite{hasselmoRoleAcetylcholineLearning2006, palacios-filardoAcetylcholinePrioritisesDirect2021,
fuchsbergerModulationHippocampalPlasticity2022}.


% --- computational models ---

\subsection{Computational models}

There have been several efforts for devising computational models capturing the construction and functionality of cognitive maps.
In relation to purely spatial navigation, a notable modelling framework is the one proposed by \cite{poucetSpatialCognitiveMaps1993}, in which the hippocampus was envisoned to encode positions with directional information, while metric properties were brought by the pariental cortex.
Later work has extended it into a route-based formalism \cite{wernerModellingNavigationalKnowledge2000}, centered on the idea of a route graph formed by concatenation of episodic routes. Further, these graphs could have different granularity, organized hierarchially in layers.
Another direction has seen the use of the successor representation (SR) algorithm, based on predictive modeling, for learning navigation tasks in spatial and semantic spaces, with good performance \cite{stoewerNeuralNetworkBased2023}, also in comparison to humans and rats \cite{decothiPredictiveMapsRats2022}.
A similar approach for generating task-dependent is the Tolman-Eichenbaum Machine (TEM), based on artificial neural networks \cite{whittingtonTolmanEichenbaumMachineUnifying2020}, which has been applied to navigation and relational reasoning.
Interestingly, some of these artificial neurons resembled the firing pattern of actual hippocampal and entorhinal cells.
Such bio-realistic neuronal tunings have also been consistently obtained in recurrent neural networks trained to solve \textit{path integration} problems, using only idiothetic information such as the agent's velocity vector \cite{baninoVectorbasedNavigationUsing2018, sorscherUnifiedTheoryOrigin2019, cuevaEmergenceGridlikeRepresentations2018}.
This line of normative research has been particularly successful in generating a rich ecosystem of spatial cells, such as grid cells and place cells, and reasoning through their functional importance for navigation.

% --- problem statement ---
However, what these models have in common is the reliance on backpropagation for training, which is not biologically plausible, and the lack of a proper real-time learning dynamics.
Although they rely solely on locally available information during inference, the type of cognitive map they can support is limited to purely spatial data. For more diffult behaviours such as goal-directed navigation, a richer representation of the environment should be more advantageous.
Furthermore, the role of neuromodulation in the formation of cognitive maps has been largely overlooked, despite its relevance in actively regulating neuronal dynamics and gating important information.

\hfill \break

% --- proposed solution ---
In this study, we introduced a model architecture capable of generating online a spatial representation made of two place cells layers, which can be endowed with additional behaviourally-relevan sensory inputs by means of neuromodulators.
Our primary goal was to successfully construct a bio-realistic cognitive map and the effectivess of neuromodulation in enriching it with sensorial information, and evaluating their impact on the agent's performance in goal-directed navigation.
One important modeling assumption was to solely rely on the spatial information available within the local graph, namely without invoking a global coordinate system, justified by the controversial evidence for a proper metric space in brain cognitive maps.
This is thus more aligned with the \textit{labelel graph} approach, although metric data is not explicitly attached to the nodes but rather inferred on-the-fly with an implicit and weak embedding into an affine space, similarly to other proposals \cite{baumannMetricInformationCognitive2023}.
Another assumption was that learning occurs entirely online, as animals do, avoiding a costly network training, which is instead popular in standard deep and reinforcement learning as well as in most models applied to \textit{path integration} \cite{baninoVectorbasedNavigationUsing2018, sorscherUnifiedTheoryOrigin2019, cuevaEmergenceGridlikeRepresentations2018}.
This feature is achieved by introducting a structural inductive bias in the form of a pre-defined stack of grid cells layers with different spatial frequencies, from which multiple layers of place cells are generated through network competition \cite{almePlaceCellsHippocampus2014, ormondPlaceFieldExpansion2015}.
A map is then obtained by connecting neighboring place cells obtained over several trajectories, thus going beyond route learning \cite{wernerModellingNavigationalKnowledge2000} and constructing a topological graph.
For what concerns neuromodulators, in this work they are best described by nodes working as analog sensors of external events. Their role is to form synaptic connections with the place cells, effectively determing a scalar field over their representation.

A secondary goal was to investigate the possibility of updating the cognitive map to environmental changes, accounting for the adaptation and flexibility in animals behaviour.
For this scope, the model was endowed with a way to correct the synaptic weights between neuromodulators and place cells, based on prediction error of future sensory observations.
This mechanism is aligned with principles of neural predictive coding \cite{schultzNeuralSubstratePrediction1997, aliPredictiveCodingConsequence2021, bonoLearningPredictiveCognitive2023, sheynikhovichLongtermMemorySynaptic2023}, according to which prediction-based learning is a relevant part of brain activity.
We took inspiration from previous work which has applied neuromodulation in spiking neurons for controlling explorative-exploitative behaviour \cite{brzoskoRetroactiveModulationSpike2015, brzoskoSequentialNeuromodulationHebbian2017}.
Further, it is well established the role of modulators such as dopamine in actively maintaining and updating the state of neural representations, particularly those associated with reward features \cite{schultzDopamineRewardPrediction2016, inglisModulationDopamineAdaptive2021, toblerAdaptiveCodingReward2005, coolsChemistryAdaptiveMind2019, decothiPredictiveMapsRats2022}.

% While others used it during training of deep artificial networks \cite{meiEffectsNeuromodulationinspiredMechanisms2023}.

% We endowed an artificial agent with this model and simple heuristics for toggling between exploration and exploitation, and tested it in a goal-directed navigation task in environments with different layout.

A last interest of ours regarded the effect of neuromodulation on the density of place cells, and in particular the reshaping of their place field, with respect to task performance.
The motivation stems in the experimental observations of modulation of place cells representation following meaningful events, in terms of remapping, fields resizing and dislocation on the map \cite{bittnerBehavioralTimeScale2017, milsteinBidirectionalSynapticPlasticity2021, fentonRemappingRevisitedHow2024}.


% Finally, the result is our operationalization of a cognitive map, a neural structure built online through experience and supporting the necessary information for behaviour.

% Our working hypothesis was that a non-uniform distribution of place cells would be beneficial for the agent's performance, especially in terms of more stable representations of special locations.


\hfill \break
The rest of the paper is organized as follows. In Section 2, we describe the model architecture and the task. In Section 3, we present the results of the simulations. In Section 4, we discuss the implications of the results, suggest future directions for research, and make conclusive observations.



