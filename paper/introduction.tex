\section{Introduction}

% BRIEF INTRO
Survival in complex environments requires efficient navigational strategies.
From desert ants to humans, successful wayfinding—navigating toward goals that are not directly visible depends on emergent internal spatial representations, known as cognitive maps \cite{golledge2000, epstein2014}.
Understanding how these maps are constructed from ongoing experiences and how they can be exploited for flexible goal-directed navigation remains an active area of research in both neuroscience and reinforcement learning (RL).

% NEUROANATOMY: hippocampal formation
The hippocampus (HP) and the entorhinal cortex (EC) serve as the main neural substrates for spatial representation in the brain.
They contain specialized neurons that encode spatial and contextual information, including grid, border, speed, and place cells \cite{sargolini2006, kropff2015, solstad2006}.
Place cells in the CA1 hippocampal region have attracted particular interest due to the convergence of inputs from entorhinal grid cells, the CA3 hippocampal region, and the lateral EC \cite{bush2014, neher2017, li2019, bilash2023}.
This strategic integration of diverse spatial and contextual signals suggests that CA1 place cells may play a critical role in the formation and maintenance of cognitive maps \cite{donato2023}.

% COGNITIVE MAP THEORIES
Traditional theories of cognitive maps suggest that spatial representations can emerge from multiple navigational strategies.
One of these is path integration, which involves tracking one’s position by integrating past movements—using action or velocity vectors—to estimate the trajectory and return to the point of origin.

An important strength of path integration lies in its independence from external cues, instead relying on idiothetic signals—internal motion cues—considered biologically plausible and thought to involve entorhinal grid cells \cite{whishaw1999, mcnaughton2006}.
A concrete implementation of this strategy is route learning, in which sequences of actions and positions are encoded along traveled paths. However, this approach scales poorly, especially at intersections or when routes overlap \cite{peer2021, chrastil2014, werner2000}.
In contrast, survey maps rely on Euclidean geometry, offering more flexible navigation capabilities \cite{chrastil2014, gallistel1996}.
Yet, their geometric assumptions often conflict with neural and behavioral data, which instead show distortions and topological biases in spatial representations \cite{peer2024, warren2019, wagner2008, rothkegel1998}.
A compromise is offered by labeled graph representations, which encode landmarks and transitions in a topological network. These support vector-like operations, planning, and prediction while avoiding strict geometric constraints \cite{meilinger2008, wang2017, schinazi2013}.

% COMPUTATIONAL MODELS I
On the computational side, various models have attempted to formalize and replicate these navigational strategies.
Early frameworks suggested that the hippocampus encodes both spatial location and direction \cite{poucet1993}, while graph-based models captured structural aspects of spatial organization, though often at the cost of scalability \cite{werner2000}.
A number of studies focused more directly on path integration, typically using recurrent neural networks combined with linear readouts. These systems, when trained on navigation tasks, have spontaneously developed spatially-tuned activity patterns reminiscent of grid, place, and border cells \cite{banino2018, sorscher2023, cueva2018}.

% Other proposals drew inspiration from predictive coding and reinforcement learning, including the successor representation algorithm (SR) \cite{stachenfeld2017} and the Tolman-Eichenbaum Machine (TEM), which generalize across spatial and relational tasks while mimicking biological neural activity patterns \cite{stoewer2023, decothi2022, whittington2020}.
% As most computational approaches, these models possess simplified neuronal dynamics, often reduced to artificial neurons defined as a weighted sum and a fixed non-linear activation function.
% Learning occurs through gradient descent over many episodes or epochs, far removed from the adaptive and context-dependent learning seen in biological systems.

Another proposal is the Tolman-Eichenbaum Machine (TEM), which generalize across spatial and relational tasks while mimicking biological neural activity patterns \cite{stoewer2023, decothi2022, whittington2020}.
Learning occurs through gradient descent over many episodes or epochs, far away from adaptive and context-dependent learning seen in biological systems.
Nevertheless, as most computational approaches these models possess simplified neuronal dynamics, often reduced to artificial neurons defined as a weighted sum and a fixed non-linear activation function.

An alternative is the successor representation framework, built around the learning of a predictive map of the explored state space, which can be used for active navigation \cite{dayan1993, blier2021}.
It has been associated with the hippocampus for the generation of a cognitive map, using the possibility of incorporating reward information and formulation in terms of biologically plausible mechanisms \cite{gardner2018, lee2022b, bono2023, fang2023}.

% Another important component are neuromodulators.
% Their actions include modulation of neuronal dynamics, for example, adjusting synaptic strength, tuning place fields \cite{lisman2005, duszkiewicz2019, schultz1997}.
% In addition, neuromodulators such as dopamine can gate reward signals that reshape the setting of places cells \cite{kempadoo2016, retailleau2014, bittner2017, kaufman2020}, support novelty detection \cite{duszkiewicz2019}, and encode prediction errors \cite{sheynikhovich2023, schultz1997}, particularly via LEC inputs \cite{igarashi2014, ito2012} - mechanisms closely related to reinforcement learning principles.

% NEUROMODULATORS
Another relevant element in brain dynamics are neuromodulators, endogenous molecules with a range of specialized actions that affect physiology, cognition, and behavior \cite{nadim2014}.
Their functions include filtering meaningful internal and external signals, influencing neuronal dynamics, and learning by altering synaptic states and parameters \cite{lisman2005, duszkiewicz2019, schultz1997}.
An important neuromodulator is dopamine, long associated with reward information, encoding of prediction errors \cite{sheynikhovich2023, schultz1997} and novelty detection \cite{duszkiewicz2019}, mechanisms closely related to reinforcement learning principles \cite{avery2017, sutton1998}.
In addition, projections from the Ventral Tegmental Area (VTA) and Locus Coeruleus (LC) have been shown to target the hippocampal circuits \cite{fuchsberger2022, edelmann2018, chowdhury2022}, to be involved in memory formation \cite{takeuchi2016}, and affect the adjustment of CA1 places cells \cite{kempadoo2016, retailleau2014, bittner2017, kaufman2020, retailleau2014}, particularly through LEC inputs \cite{igarashi2014, ito2012}.

% COMPUTATIONAL MODELS
Others incorporate reward-driven Hebbian plasticity modulated by neuromodulators \cite{brzosko2019}.
Nevertheless, these architectures mostly fail to unify these ingredients into a biologically grounded system that at the same time learns a map of the environment online without relying on an external coordinate system, and flexibly performs goal-directed navigation.

% OUR WORK
In this work, we present a biologically inspired model of cognitive map formation that integrates grid cells, place cell representations, neuromodulatory signals, and graph-based spatial computations.
We first aim is to demonstrate that an agent endowed with such bioinspired model architecture is capable of building content-rich topological maps of the surroundings on the fly, and leveraging it for efficient goal-directed navigation.
The testing protocol was a common behavioural and reinforcement learning task: exploration of a novel enviroment, and the localization and collection of a reward.
Further validation was performed in multiple experiments in which some aspects of the protocol were varied.

% place cell map
For building the cognitive map, a strategy similar to path integration has been used in that the agent develops a place cells representation as it navigates in unexplored spaces.
However, spatial tuning is achieved through fast synaptic plasticity and competition, and not through numerous backpropagation cycles.
The use of idiothetic information only and not other sensory modalities, such as visual cues, is motivated by reducing architectural complexity and proving the validity of the approach with minimal sources of information.

% neuromodulation
Concerning neuromodulation, it has been operationalized as a collection of variables acting as low-pass filters of sensory events \cite{ali2021, sheynikhovich2023, bono2023}, here being only rewards and collisions with boundaries.
The action of neuromodulators is to form adaptive connections with place cells through local Hebbian plasticity, effectively determining a scalar field over the spatial map \cite{sosa2024} that can be used for planning \cite{schultz2016, inglis2021, tobler2005, cools2019, decothi2022}.
Moreover, they were also given the opportunity to directly modulate some neuronal properties of place cells, such as the gain of neural activation and the place field.

In fact, another objective of ours was to evaluate how active neuromodulation of neuronal dynamics can impact task performance.
To this end, multiple comparative ablation experiments were performed to highlight the relevance of each modulatory mechanism.

Finally, we demonstrate how the model dynamically adapts to environmental changes and how some neuromodulatory actions can positively influence the formation and flexibility of a cognitive map \cite{milstein2021, fenton2024}.

% ORGANIZATION
The remainder of the paper is organized as follows: Section 2 details the model and experimental setup; Section 3 presents results; Section 4 discusses broader implications and future directions.

