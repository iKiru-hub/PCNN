
\section{Discussion}
Exploration and planning in known and past environment are essential behaviours of animals, directly affecting their success in world udnerstanding and goal reaching.

% brief recap of the background
An important element behind these abilities is the formation of a map of their surroundings as they make new experiences, known as a cognitive map.
The hippocampus, and especially CA1, is one of the principal brain areas involved in this processes \cite{donatoHowYouBuild2023}.
Numerous speculations have been made about the shape and neural foundations of such object, varying in the type of modeling assumptions and experimental support.
In terms of the map structure, an important aspect is the rigidity of the metric constraints, which concerns the type of operations that can be applied to the underlying spatial representations \cite{gallistelComputationsMetricMaps1996, chrastilCognitiveMapsCognitive2014, warrenNonEuclideanNavigation2019}.
Regarding its formation, computational approaches relying on training artificial neural networks have been used for reproducing experimental observations, such as path integration \cite{baninoVectorbasedNavigationUsing2018, cuevaEmergenceGridlikeRepresentations2018, sorscherUnifiedTheoryOrigin2019, whittingtonTolmanEichenbaumMachineUnifying2020} and navigation \cite{poucetSpatialCognitiveMaps1993, decothiPredictiveMapsRats2022}.
Another important ingredient for neuronal dynamics is neuromodulation.
Its implications for spatial cognition have been in part associated to the role of dopamine, especially in the context of plasticity, predictive learning, and reward representation \cite{kempadooDopamineReleaseLocus2016, duszkiewiczNoveltyDopaminergicModulation2019, schultzDopamineRewardPrediction2016}.

% brief recap of this work
The objective of the present work was to devise a network model, inspired by the CA1 region, embedded into an agent exploring and reaching target position online in different enviornments.
We used grid cells, together with synaptic plasticity for developing context-rich maps based on place cells, and updating them through experience.
In the spirit of minimizing the geometric assumptions on the neural space, we treated it as a topological graph with information added locally through the action neuromodulators. This aligned with the concept of a \textit{labeled graph} \cite{ishikawaSpatialKnowledgeAcquisition2006, warrenNonEuclideanNavigation2019}.
For the practical utilization of the map, graph based algorithms were used that rely on local spatial measures.

% brief recap of the results


% steelman the findings

% This is motivated by the consolidated phenomenon of hippocampal rate remapping, for which the place cells change their firing pattern according to contextual shifts \cite{andersonHeterogeneousModulationPlace2003, fentonRemappingRevisitedHow2024}. Additionally, there is growing experimental evidence that place fields can be moved in space following
% behaviourally relevant events, such as the occurrence of reward, according to a plasticity rule known as behavioural time-scale plasticity (BTSP) \cite{bittnerBehavioralTimeScale2017, miikkulainenEvolvingDeepNeural2017}.
% In our model, we associated this process to both collision and reward signals, whose location is set to be the center towards which the cells within a certain radius $r_{\text{BND}},\;r_{\text{DA}}$ are pulled. The centers of the cells involved are shifted with a force proportional to the Gaussian distance from the center of the signal, and the strength is weighted by a parameter $\lambda_{\text{BND}},\;\lambda_{\text{DA}}$.

% show limitations


% conclusion
