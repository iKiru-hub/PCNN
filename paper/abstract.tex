\begin{abstract}

% the context
During navigation, animals dynamically create rich representations of the environment, forming personalized cognitive maps used for exploration and goal planning. The hippocampal area CA1 features spatial cells that adapt based on behavior and internal states.
A possible modeling approach is a labeled graphs that, with the intent of avoiding a map with metric structure, relies on nodes enriched with spatial information only specified locally.
Another popular direction if training of deep neural networks on spatial tasks, from which record network dynamics from emerging spatially tuned neurons.

% the model in brief
In this study, we introduce a place-cells based architecture for developing cognitive maps in one-shot while exploring novel environments.
We used a simulated agent for reward-driven navigation tasks, which operates online and forms spatial representations of its surroundings.
Further, by means of neuromodulators it incorporates behaviorally relevant information, such as boundaries and reward location.
Learning was involved the combination of a rapid Hebbian plasticity, lateral competition, and modulation of the place cells,

% results
The agent proved successful in exploring, retrieving and reaching goal locations in a variety of environments, and displayed adaptability when the reward was moved.
Further, the analysis of the neuromodulated place cells showed the importance of dynamically changing neuronal density and tuning field size following relevant events.
These results align with experimental evidence of reward effects on hippocamapal spatial cells, and provides additional computational support to the labeled graph approach.

\end{abstract}
