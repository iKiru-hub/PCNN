\section{Introduction}
\hfill \break
\vspace {0.5cm}


% --- catchy intro ---
Survival in complex environments requires adequate agency abilities, namely the capacity to explore, learn, and navigate toward goals. This becomes especially challenging when destinations are out of sight, a scenario known as \textit{wayfinding}, where effective behavior relies on an internal model of the world \cite{golledgeCognitiveMapsSpatial2000, epsteinNeuralSystemsLandmarkbased2014}.

To support such behavior, the mammalian brain has evolved powerful spatial representations of the environment, refined through repeated experiences. These internal representations—commonly known as \textit{cognitive maps}—were first proposed by Tolman, who observed that rats could navigate mazes in a goal-driven fashion, even when typical cues were removed \cite{tolmanCOGNITIVEMAPSRATS1948}.

% --- cognitive map and function ---

Several strategies have been proposed to explain how agents reach known targets using cognitive maps.
A basic method is \textit{route learning}, where paths are stored as action-position pairs.
Though effective in small-scale settings, it struggles with scalability due to ambiguities at path intersections \cite{peerStructuringKnowledgeCognitive2021, chrastilCognitiveMapsCognitive2014, wernerModellingNavigationalKnowledge2000}.
A more flexible alternative is \textit{survey knowledge}, involving metric representations that support vector operations in Euclidean space \cite{chrastilCognitiveMapsCognitive2014, gallistelComputationsMetricMaps1996}.
However, empirical evidence suggests real brain maps often violate strict geometric constraints \cite{peerFormatCognitiveMap2024, warrenNonEuclideanNavigation2019, wagnerComparingPsychophysicalGeometric2008, rothkegelJudgingSpatialRelations1998}.
A compelling compromise is the \textit{labeled graph} model: a topological network where spatial relations are encoded locally through experience-based labels \cite{meilingerNetworkReferenceFrames2008, chrastilCognitiveMapsCognitive2014, wangLearningReinforcementLearn2017, ishikawaSpatialKnowledgeAcquisition2006}.
This structure tolerates global inconsistencies while supporting local vector-like operations, aligning with both behavioral studies \cite{byrneMemoryUrbanGeography1979, warrenNonEuclideanNavigation2019, schinaziHippocampalSizePredicts2013} and \textit{path integration} mechanisms observed across species \cite{wehnerVisualNavigationInsects1996, gillnerNavigationAcquisitionSpatial1998, gallinaroSynapticWeightsThat2023}.

The neural substrates behind cognitive maps have been predominatly traces to the hippocampus and its surroundings, where arious neuronal types have been linked to allocentric and egocentric spatial features.
Border cells signal environmental boundaries; speed cells encode locomotion velocity; head-direction cells track angular orientation; place cells respond to specific locations; and grid cells exhibit hexagonal spatial tuning \cite{sargoliniConjunctiveRepresentationPosition2006, kropffSpeedCellsMedial2015, solstadGridCellsPlace2006}.
These cells are primarily found in the entorhinal cortex (especially medial EC) and hippocampal formation, including areas CA1, CA3, and the subiculum.
Place cells in CA1, in particular, are often viewed as key components of cognitive maps \cite{donatoHowYouBuild2023}.
The origin of their spatial tuning remains debated, with hypotheses focusing on input from entorhinal grid cells via the temporo-ammonic pathway or recurrent projections from CA3 through the Schaffer collaterals \cite{bushWhatGridCells2014, neherGridCellsPlace2017a, liModelingPlaceCells2019, kubieSpatialFrequenciesGrid2015}.

% --- cognitive map and neuromodulation ---
Place cells in CA1 integrate spatial input from the medial EC and CA3, along with sensory and contextual signals from the lateral EC \cite{bilashLateralEntorhinalCortex2023}. These inputs are modulated by neuromodulators which dynamically gate information and influence plasticity.
In this regard, a primary role is covered by the dopaminergic projections from the VTA and LC target hippocampal subregions, including CA1, where they modulate excitatory input, regulate long-term potentiation (LTP) via D1-like receptors, and influence memory consolidation \cite{lismanHippocampalVTALoopControlling2005, tsetsenisActivationLocusCoeruleus2022, zhangExpectancyrelatedChangesFiring2024, edelmannDopaminergicInnervationModulation2018, wagatsumaLocusCoeruleusInput2018}.
Dopamine conveys reward-related signals that reshape place cell tuning \cite{kempadooDopamineReleaseLocus2016, retailleauMichelinRedGuide2014, bittnerBehavioralTimeScale2017, kaufmanRoleLocusCoeruleus2020}, supports novelty detection \cite{duszkiewiczNoveltyDopaminergicModulation2019}, and encodes reward prediction errors \cite{sheynikhovichLongtermMemorySynaptic2023, schultzNeuralSubstratePrediction1997}, especially in relation to LEC inputs \cite{igarashiFunctionalDiversityTransverse2014, itoFunctionalDivisionHippocampal2012}.

\hfill \break
Computational models of cognitive maps have evolved from early proposals—where the hippocampus encodes position and direction, and the parietal cortex supplies metric information \cite{poucetSpatialCognitiveMaps1993}—to route-based topological graphs \cite{wernerModellingNavigationalKnowledge2000}. 
Recent models leverage predictive coding, such as the successor representation and Tolman-Eichenbaum Machine, which generalize across spatial and relational tasks while mimicking biological activity \cite{stoewerNeuralNetworkBased2023, decothiPredictiveMapsRats2022, whittingtonTolmanEichenbaumMachineUnifying2020}. 
Path integration models trained on motion cues also generate grid- and place-like tunings \cite{baninoVectorbasedNavigationUsing2018, sorscherUnifiedTheoryOrigin2019, cuevaEmergenceGridlikeRepresentations2018}, and some incorporate neuromodulation-driven Hebbian plasticity for reward-based learning \cite{brzoskoNeuromodulationSpikeTimingDependentPlasticity2019}.

Although many of these frameworks feature bio-realistic tuning, predictive learning, or neuromodulation, none integrates all elements simultaneously, and most depend on biologically implausible training like backpropagation.


% --- proposed solution ---
\hfill \break
% snapshot
In this study, we present a model that builds an online spatial representation of place cells, enriched with behaviorally relevant sensory information through neuromodulators.

The cognitive map is represented as a topological graph, where neighboring place cells are linked across experienced trajectories, going beyond traditional route learning \cite{wernerModellingNavigationalKnowledge2000}. Goal-directed navigation is then achieved through a path-finding algorithm that operates on this graph-level spatial data, aligning with the \textit{labeled graph} framework \cite{baumannMetricInformationCognitive2023}.

Neuromodulators are modeled as analog sensor nodes that form synapses with place cells, defining scalar fields over the map \cite{sosaHippocampalSequencesSpan2024}. Learning is conducted entirely online, mimicking animal behavior and avoiding the costly training processes typical of deep and reinforcement learning models, particularly in path integration tasks \cite{baninoVectorbasedNavigationUsing2018, sorscherUnifiedTheoryOrigin2019, cuevaEmergenceGridlikeRepresentations2018}. This is enabled by an architectural inductive bias: a predefined stack of grid cell layers with varying spatial frequencies, from which competing networks generate place cells \cite{almePlaceCellsHippocampus2014, ormondPlaceFieldExpansion2015}.

Our primary goal is to validate the emergence of a cognitive map that reflects the agent's experiences and to assess its effectiveness in supporting goal-directed navigation. A secondary objective is to explore the map's adaptability to environmental changes. To achieve this, we introduce a synaptic update mechanism based on prediction errors in future sensory inputs, inspired by predictive coding theories \cite{schultzNeuralSubstratePrediction1997, aliPredictiveCodingConsequence2021, bonoLearningPredictiveCognitive2023, sheynikhovichLongtermMemorySynaptic2023}. This aligns with prior work on neuromodulation in spiking networks for exploration-exploitation dynamics \cite{brzoskoRetroactiveModulationSpike2015, brzoskoSequentialNeuromodulationHebbian2017}, and supports the view that modulators like dopamine are essential for maintaining and adapting neural representations tied to rewards \cite{schultzDopamineRewardPrediction2016, inglisModulationDopamineAdaptive2021, toblerAdaptiveCodingReward2005, coolsChemistryAdaptiveMind2019, decothiPredictiveMapsRats2022}.

Finally, we investigate how neuromodulation affects the density and spatial distribution of place fields, driven by experimental observations of remapping, resizing, and relocation in response to significant events \cite{bittnerBehavioralTimeScale2017, milsteinBidirectionalSynapticPlasticity2021, fentonRemappingRevisitedHow2024}.


\hfill \break
The rest of the paper is organized as follows. In Section 2, we describe the model architecture and the task. In Section 3, we present the results of the simulations. In Section 4, we discuss the implications of the results, suggest future directions for research, and make conclusive observations.

