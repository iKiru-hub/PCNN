\section{Introduction}
\hfill \break
\vspace {0.5cm}


% --- catchy intro
Survival in complex environments requires adequate navigational abilities, including exploration, charting, and planning.
Such task is especially challenging when destinations are out of sight, a scenario known as \textit{wayfinding}, where the reliance on an internal world map becomes essential \cite{golledgeCognitiveMapsSpatial2000, epsteinNeuralSystemsLandmarkbased2014}.

% --- cognitive maps
Cognitive map theories propose multiple strategies for spatial navigation, from simple route learning to survey knowledge and labeled graph models. Route learning stores paths as action-position pairs but struggles with scalability at intersections \cite{peerStructuringKnowledgeCognitive2021, chrastilCognitiveMapsCognitive2014, wernerModellingNavigationalKnowledge2000}. Survey knowledge offers flexibility through metric representations in Euclidean space \cite{chrastilCognitiveMapsCognitive2014, gallistelComputationsMetricMaps1996}, though empirical evidence shows brain maps often violate geometric constraints \cite{peerFormatCognitiveMap2024, warrenNonEuclideanNavigation2019, wagnerComparingPsychophysicalGeometric2008, rothkegelJudgingSpatialRelations1998}. The labeled graph model provides a compelling compromise—a topological network with experience-based local encodings \cite{meilingerNetworkReferenceFrames2008, chrastilCognitiveMapsCognitive2014, wangLearningReinforcementLearn2017, ishikawaSpatialKnowledgeAcquisition2006} that tolerates global inconsistencies while supporting vector operations, aligning with behavioral studies \cite{byrneMemoryUrbanGeography1979, warrenNonEuclideanNavigation2019, schinaziHippocampalSizePredicts2013} and path integration mechanisms across species \cite{wehnerVisualNavigationInsects1996, gillnerNavigationAcquisitionSpatial1998, gallinaroSynapticWeightsThat2023}.

% --- spatial cells
Neural substrates underlying cognitive maps primarily involve the hippocampus (HP) and entorhinal cortex (EC), hosting specialized cells that encode spatial features: border cells signaling environmental boundaries; speed cells encoding locomotion velocity; grid cells exhibiting hexagonal spatial tuning; and place cells responding to specific locations; \cite{sargoliniConjunctiveRepresentationPosition2006, kropffSpeedCellsMedial2015, solstadGridCellsPlace2006}.
The latter are found abundandly in hippocampal area CA1, and are key components of cognitive maps \cite{donatoHowYouBuild2023}, with their spatial tuning potentially emerging from entorhinal grid cells via the temporo-ammonic pathway or CA3 projections through Schaffer collaterals \cite{bushWhatGridCells2014, neherGridCellsPlace2017a, liModelingPlaceCells2019, kubieSpatialFrequenciesGrid2015}.

% --- neuromodulation
Dopaminergic neuromodulation dynamically influences cognitive maps. CA1 place cells integrate spatial input from medial EC and CA3, plus sensory and contextual signals from lateral EC (LEC) \cite{bilashLateralEntorhinalCortex2023}, all modulated by dopaminergic projections from VTA and LC that regulate excitatory input and plasticity \cite{lismanHippocampalVTALoopControlling2005, tsetsenisActivationLocusCoeruleus2022, zhangExpectancyrelatedChangesFiring2024, edelmannDopaminergicInnervationModulation2018, wagatsumaLocusCoeruleusInput2018}.
These projections transmit reward signals reshaping place cell tuning \cite{kempadooDopamineReleaseLocus2016, retailleauMichelinRedGuide2014, bittnerBehavioralTimeScale2017, kaufmanRoleLocusCoeruleus2020}, support novelty detection \cite{duszkiewiczNoveltyDopaminergicModulation2019}, and encode prediction errors \cite{sheynikhovichLongtermMemorySynaptic2023, schultzNeuralSubstratePrediction1997}, particularly via LEC inputs \cite{igarashiFunctionalDiversityTransverse2014, itoFunctionalDivisionHippocampal2012}—mechanisms closely related to reinforcement learning principles.

% --- computational models
Computational models of cognitive maps have evolved significantly, paralleling developments in reinforcement learning (RL).
From early proposals where hippocampus encodes position and direction \cite{poucetSpatialCognitiveMaps1993} to route-based topological graphs \cite{wernerModellingNavigationalKnowledge2000}, recent models leverage predictive coding frameworks—including successor representation and Tolman-Eichenbaum Machine—that generalize across spatial and relational tasks while mimicking biological activity patterns \cite{stoewerNeuralNetworkBased2023, decothiPredictiveMapsRats2022, whittingtonTolmanEichenbaumMachineUnifying2020}.
Path integration models trained on motion cues generate grid- and place-like tunings \cite{baninoVectorbasedNavigationUsing2018, sorscherUnifiedTheoryOrigin2019, cuevaEmergenceGridlikeRepresentations2018}, while others incorporate neuromodulation-driven Hebbian plasticity for reward-based learning \cite{brzoskoNeuromodulationSpikeTimingDependentPlasticity2019}.
Despite these advances in bio-realistic tuning, predictive learning, and neuromodulation modeling—all elements critical to RL approaches—current frameworks fail to integrate these components simultaneously within biologically plausible learning paradigms, also often requiring extensing offline training.

% --- propsed solution
In this study, we present a model that builds an online spatial representation using place cells enriched with behaviorally relevant sensory information through neuromodulation.
Our cognitive map forms a topological graph where neighboring place cells link across experienced trajectories, transcending traditional route learning \cite{wernerModellingNavigationalKnowledge2000}.
Goal-directed navigation is achieved through path-finding algorithms operating at the graph level, aligning with the labeled graph framework \cite{baumannMetricInformationCognitive2023}.

Neuromodulators are modeled as analog sensor nodes forming synapses with place cells, defining scalar fields across the map \cite{sosaHippocampalSequencesSpan2024}.
Learning occurs entirely online, mimicking natural behavior and avoiding costly training processes typical in deep- and reinforcement-learning models for path integration \cite{baninoVectorbasedNavigationUsing2018, sorscherUnifiedTheoryOrigin2019, cuevaEmergenceGridlikeRepresentations2018}.
This efficiency stems from an architectural inductive bias: a predefined stack of grid cell layers with varying spatial frequencies from which competing networks generate place cells \cite{almePlaceCellsHippocampus2014, ormondPlaceFieldExpansion2015}.

Our primary objective is validating the emergence of a cognitive map reflecting agent experiences and assessing its effectiveness in goal-directed navigation.
We also explore map adaptability to environmental changes through a synaptic update mechanism based on prediction errors in sensory inputs, inspired by predictive coding theories \cite{schultzNeuralSubstratePrediction1997, aliPredictiveCodingConsequence2021, bonoLearningPredictiveCognitive2023, sheynikhovichLongtermMemorySynaptic2023}.
This approach aligns with research on neuromodulation in spiking networks for exploration-exploitation dynamics \cite{brzoskoRetroactiveModulationSpike2015, brzoskoSequentialNeuromodulationHebbian2017} and supports the view that modulators like dopamine maintain and adapt neural representations tied to rewards \cite{schultzDopamineRewardPrediction2016, inglisModulationDopamineAdaptive2021, toblerAdaptiveCodingReward2005, coolsChemistryAdaptiveMind2019, decothiPredictiveMapsRats2022}.

Finally, we investigate how neuromodulation affects place field density and distribution, motivated by experimental observations of remapping, resizing, and relocation in response to significant events \cite{bittnerBehavioralTimeScale2017, milsteinBidirectionalSynapticPlasticity2021, fentonRemappingRevisitedHow2024}.

% --- toc
The remainder of the paper is organized as follows. Section 2 describes the model architecture and task. Section 3 presents simulation results. Section 4 discusses implications, future research directions, and concluding observations.

