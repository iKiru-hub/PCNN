\section{Results}

\begin{figure}[h]
    \centering
    \includegraphics[width=0.7\textwidth]{figures/collda.png}
    \caption{\textbf{Cognitive maps and performance results} -
    \textbf{a}: \textit{a cognitive map over a space, together with the plan (red line) to reach a target location from a starting position.}.
    \textbf{b}: \textit{the same enviroment but with the reward (green circle), trajectory (red line), agent position (black square).} -
    \textbf{c}: \textit{plot of trajectories before (black) and after (red) the insertion of a wall (rectangle) between the starting and goal positions, the wall can also be spotted from the boundary cells in blue} -
    \textbf{d}: \textit{trajectories for multiple trials with the agent starting at the same position (black square) but with the reward location (green circles) periodically moving} -
    \textbf{e}: \textit{place cells centers with size proportional to their node degree} -
    \textbf{f}: \textit{place fields of the same cell before and after several relocation of its center following reward events} -
    \textbf{g}: \textit{performance comparison for the same models on the same environment with and without modulation enabled. Pair-wise t-test over 128 iterations and Bonferroni correction} -
    \textbf{i}: \textit{effect of reward count on reward and boundary modulated cells (green and blue respectively), both in count (top row) and gain modulation magnitude (bottom row).} - \textbf{l}: \textit{same but with respect to collision count} - \textbf{m}: \textit{relation between count of reward and
boundary modulated cells, and between gain modulation magnitude}}
    \label{fig:main_results}
\end{figure}


\paragraph{Performance in wayfinding}
Our primary aim was to evaluate the formation of the cognitive map through neuromodulation in terms of the performance of the goal navigation in different environments.
The best model resulting from evolution reached solid navigation and adaptation skills.
The agent was able to visit a significant portion of the environment during exploration and use neuromodulation to produce useful spatial representations.

The left panel of plot \ref{fig:main_results}\textbf{a} displays place cells associated with collisions and reward events, signaling boundaries (in blue) and reward (in green) locations.
The overlap of these two representations and the place cells (in pink) is what we refer to as a cognitive map, since these are the main sources of spatial and contextual information used during planned navigation, whose path is depicted as a gray line.
The right panel instead portrays the actual environment with walls (black), reward location (green), and multiple trajectories (red).
During exploration, the main areas were visited until the reward position was located and the goal-directed navigation dominated, as highlighted by the density of the path lines.
Considering the position of the walls and corners, the layout of this environment does not always make the target locations visible, as it is a non-convex area and therefore can be classified as wayfinding \cite{meilinger2016}.
The challenge of not being able to use straight lines is overcome by the graph approach using local data and the consideration of boundary place cells, allowing the agent to plan accordingly.
In addition, by construction the path also minimized the length of the path, within the part of the space covered by the cognitive map.

In general, this result confirms the ability of the model to focus on navigation and obstacle avoidance.
However, it is worth nothing that not all simulations resulted in a reward being found in the first place, due to the randomness of the exploratory process; this was more pronounced in complex environment.

\paragraph{Detour task}
The planning ability and the plastic nature of the cognitive map should provide resilience against unexpected changes in the environment layout.
In order to verify this we implemented a detour experiment.
Initially, the agent was familiarized with a square environment with the reward in the middle and starting always from the same position.
Then, a wall was placed in between the starting position and the reward, therefore forrcing new trajectory for reaching it.
As expected, the agent was able to form a representation of the new obstacle and calculating new paths around it, succeeding the task.
In plot \ref{fig:main_results}\textbf{c} they are shown the trajectories before and after the wall placement, and it is manifesed the ability of detour in the new layout.

\paragraph{Adaptive goal representation through sensory error}
Then, we tested the adaptability to environmental changes. In this scenario, the reward object was moved after being fetched a fixed number of times.
Here, the difficulty was to unlearning previous locations and discovering new ones, in a protocol similar to \cite{brzosko2019}.
In plot \ref{fig:main_results}\textbf{d} is reported the set of trajectories over many trials with the reward displaced in three possible locations. The agent was capable of planning behavior, as earlier, but also exploring and finding the new rewards, as shown by the density of lines.
Whenever a goal path resulted in a failed prediction, the DA-based sensory error weakened the association between the place cells and the reward signal, leading to an extinction of its representation at that location.


This result validates the resilience of the model to changing sensory expectations, in this case the reward position.

\paragraph{Modulation of spatial resolution affects performance}
Lastly, we investigated the effect of modulating the density of place cells and the size of the field.
The goal position was fixed, but the agent was randomly relocated after fetching; performance was defined as the total number of reward counts within a time window.

Our working hypothesis is that these experience-driven neuronal changes would improve the quality of the cognitive map and be reflected in navigational abilities.
The assessment of this claim was conducted by comparing two variants of the model: with or without the modulation mechanisms, namely the density and gain modulation from reward and collision events..
Both models were ran in the same environment, with numerous wall, for a total of 128 simulations for each case.

The statistical results shown in the box plot of Figure \ref{fig:main_results}\textbf{g} demonstrate a signiticant difference between the two groups, supporting the important of place field modulation.

Plot \ref{fig:main_results}\textbf{e} showcases the distribution of place cells with the circle size and color represent the node degree, which aligns discretely with the reward position and the density.
Further, in plot \ref{fig:main_results}\textbf{f} the place field of one cell is shown, before and after several reward occurences and consequent center relocation.

Taken together, these findings support the hypothesis of practical utility of direct modulation of place-field structure for active navigation, even in these limited settings.




